\section{Conclusion}
% 0.5 page
At the point of writing, the project has returned to its tracks, progressing nicely according to schedule.

Our gradual approach to SCRUM have not showed any drawbacks, but have still been a useful tool to manage the current work and keep track of the progress during the entire second part of the project.

The use of Poker Planning proved to be a great way for estimating tasks - the process itself was enjoyable, and the resulting estimates proved to be good.

The dependency diagram we created proved to be an excellent companion to SCRUM as it complemented the backlog and burndown chart well: For the backlog it helped prioritizing the tasks while always showing the next upcoming task, and with the burndown chart it was able to give a hint at the current progress towards the goal.

The virtual task board we had decided to use did not give the benefit of overview as we had imagined, but still it provided a good way to track the progress of each individual task.

Our custom method for planning meeting times and keeping track of them using a simple spreadsheet also proved very effective. It was easy to use, and the additional time it reserved for unknown factors ended being put to use.
\newpage
For the product we chose to measure the quality by acceptance testing our product against its requirements.
To ensure the quality of the product we defined a testing strategy, which among others was put use through automatic unit tests. The use of peer reviewing helped detecting multiple problems early on and have thereby proven a great asset in assuring the quality of the code behind the product.