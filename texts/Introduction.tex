\section{Introduction}
% 0.5 page
This paper was written for the course `System Development and Project Organization, BSUP' at IT-University of Copenhagen in the spring 2013.

In almost every software development project some degree of planning and application of development methods are needed - a need which increases with the size of the project group, the project size, and critical factors \cite[p. 134]{ac}. For larger projects the right choices of planning and development methods are essential for meeting the requirements and deadlines.

When a project goes off its tracks, however, it can be difficult to get it back. First one must review and adjust the chosen methods used for the project to avoid further derail from happening, which involves discovering why the project derailed in the first place.
Once the work process has been adjusted, changes to the project plan are also necessary: If the right adjustments are not made, the project is unlikely to be finished with desirable quality within the designated time frame.

In this paper we consider a such derailed project.
In the first part we will empirically describe the events which lead up to the derailing, identifying the cause of the problems. In the second part we will attempt to identify the proper methods for getting the project back on track and describe the observed results of applying these. At last we will discuss and conclude how well we managed to `rescue' the project based on the resulting quality, and finally we will give a perspective of our findings.