\section{Getting back on track}
% 13 pages
We will now describe how we plan to get the project back on its tracks.

Doing this is going to take a lot of planning. However, while planning can help structure the collaboration, it can add quite a bit of overhead. Therefore, we must carefully select the extent of our planning, and the methods we wanted to use. If we use to long planning, we might never start the project, and if we use to little time planning, everything might derail again.

In the following sections we will describe the methods used to get the project back on track. Where possible we will reflect upon the results of using that particular method as well.

\subsection{Planning second half}
% 1 page
The best way to get the project back on track is through careful planning. First of all we needed to get an overview of the remaining parts of the project, and then we wanted to carefully structure and plan: What needs to be done?, In what order? and How much time do we have?

We have divided the planning up into four parts.

We start out by looking at the development methods that we want to use. We give a brief introduction to the method, and explain how we applied it, helping the project back on track

After this we describe the estimation methods we use. The estimation methods should help us make accurate estimations. This, in turn, helps us to meet deadline, as most of our estimations should be correct or close to.

Now that we have found out how much time we need, we need to find out how much time we have available. For this we use a couple of methods for planning the time. Those methods are explained, and our usage is analyzed.

As the last, but most important part, we look into how our save of the product has affected the quality. Less time to do roughly the same work must have some effect on quality. This is analyzed.
\subsection{Method of development}
% 3 pages
We like the Scrum development method, but we decided not to do full-on Scrum, but to do some Scrum activities. Daily Scrum is a great activity. When you do Daily Scrum the Development team, the Product owner and the Scrum master is held updated with progress and set-backs. We have chosen not to do Daily Scrum, because we are a small group of only 5 people, and we don't have the roles used for proper Scrum (Product owner, Scrum master, etc.). We are only the Development team and we have the API and Backlog to as tools to keep each other up to date, and there for Daily Scrum is not worth the overhead.

We chose to use another Scrum artifact: The backlog. The backlog is a list of items. Each item describes a `story', possibly with sub stories or tasks, that needs to be completed. An item on a backlog could be ``Implement controller class'' or ``Write problem statement''. Our complete backlog can be viewed in \acref{backlog}.

We chose not to divide our project into Sprints. Sprints are great for projects with a longer time span than ours, as it divides the project into smaller parts which makes it almost iterative. Sprints also function as part goals. After every Sprint, the Scrum team do a Review, do a Retrospective and plan the Backlog items for the next Sprint. Were we to divide our project into Sprints, it would be very short ones and it would not be worth the overhead.

We used Scrumwise.com as our Scrum tool. Scrum task board and burndown chart were by default given to us without any extra effort. We chose to use this "free service" to regularly check up on our progress, and to see if we had to stop and reevaluate anything - like the estimation for some or all of the tasks. It helped us gain overview doing the project. Our task board can be viewed in \acref{taskboard}. Our burndown chart can be viewed in \acref{burndown}.
\subsection{Estimation methods}
% 4 pages
\subsection{Methods for planning the time}
% 3 pages
Planning time on a project that has derailed can be considered the most important thing, although it depends on the other activities in order to be applied with success.

In the following we will describe the methods we have applied to the project, along with a single method which was considered, but was not deemed beneficial and hence not applied.

\subsubsection{Time schedule}
To be able to plan our time efficiently, we had to find out how much time we had at our disposal. Our way of doing this was to create a custom spreadsheet, containing all the dates until the final deadline. For each date a field for each group member was made, wherein the available time span for that person was entered. This way it was easy for us to get an overview of when we all were available. This helped us a lot in planning our meetings.

Generally, we decided, that if three or more group-members were available, a meeting would be planned. We would rather plan too many meetings and be done with everything early, than plan the exact hours needed to complete all the backlog items and risk time pressure due to unforeseen tasks or delays.

From the first part of the project we had learned that all-nighters - that is, a meeting continuing until the next morning - were effective ways to get a lot of work done, even though the productivity decreased over night. With this experience we converted as many pairs of subsequent days into all-nighters as we deemed it possible (see figure \ref{fig:time_schedule}).

\begin{figure}[H]
  \includegraphics[width=\textwidth]{illustrations/partialTimeSheet}
  \caption{Time schedule}
  \label{fig:time_schedule}
\end{figure}

\cref{time schedule} shows our custom spreadsheet using our custom method. We have had great benefit from doing it this way, as planning group meetings was instantaneous without too much discussion back and forth. The dates with bold time spans indicate days with meetings. The double fields with blue background indicate planned all-nighters.
\subsubsection{Dependency network with activity durations}
After planning all our group meetings we decided to create a dependency network with activity durations. We have several reasons for doing this
\begin{itemize}
	\item We wanted to make sure that everyone always knew what to do. A dependency network is great for this purpose, as everyone can easily see which elements must precede others.
	\item We wanted to clearly identify the critical path. This allow us to make sure that minimum slack is applied to the path and to keep project on deadline.
	\item To be able to see available slack of backlog items. This helped us very much in determining which items to complete first.
\end{itemize}

-- Insert diagram here --

The above network shows our dependency network.
Some important parts to point out that we learned from this network:
\begin{itemize}
	\item It shows that testing WCF controller can be pushed all the way to the 73\textsuperscript{rd} hour. We suspected that this part was not that important, but the network really confirmed us in this suspicion.
	\item It shows that we have to begin programming the controllers/models before views. It also shows that not all controllers/models need to be finished before the first views can be implemented. It shows exactly which controllers/models that needs to be finished.
\end{itemize}

The network has helped us a lot in our work, and we have returned to it many times throughout the project.
\subsubsection{GANTT diagram}
GANTT diagrams can be an extremely efficient tool for planning in larger groups. Some overhead must be expected, but it can help developers keep on track, and never not know what to do.

The overhead from using a GANTT diagram can be caused by the fact that we create an order in which the program is created. We want it to be agile enough to be able to switch between items as we like.

We had spent a lot of time constructing a thorough dependency network, and we decided to use this alone, as the base of doing items. We agreed that the dependency network provided the right overview of our items.

Because we had a project that was derailed, we feared that the overhead applied by using a GANTT diagram was too high, compared to the benefit gained from it. The benefit would also be minimal, because of our extensive dependency network.
\subsection{Quality assurance}
% 3 pages
\subsection{Summary}
First step was to plan the rest of the project. We used the SCRUM artifact Backlog to get an overview. We estimated work hours for each item on the backlog, to further enhance our overview. This gave us an estimate of the number of remaining hours. We also planned our available hours. Comparing our available working hours with the hours remaining in the project, showed us that we had just enough time, assuming our estimates rang true. ``The project is not derailed then?'' you might ask, and it was: We planned a lot more available hours in order to meet the project's remaining hours.
\newpage