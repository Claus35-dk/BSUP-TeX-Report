\section{Problem Analysis}
The derailed project we considered was our 2\textsuperscript{nd} year project, which was divided into two parts. In the first part of the project we cooperated with a Singaporean group to built a rental system. We built the server part of the system, while the Singaporean group simultaneously developed the client to consume our services.

We out fine by defining the requirements of the system, listing the functional and non-functional requirements. This helped us define what we wanted to make and the scope of it.

As development methodology we chose not to use SCRUM, since most of the group based on previous experience felt that we would spend more resources to set it up than we would gain from using it. Also our individual time schedules and short time frame made it difficult to set up sprints of effective lengths, where everybody would be able to participate. We did not anticipate how much the lack of detailed planning affected our work, but expected that the time we saved would exceed the extra spent from not planning.

From then on we had no overview of the course of the first part of the project. We decided not to define tasks and parts that needed to be done and give them estimations.
We had deadlines on when the different parts of the service had to be fully available, but the dates were set arbitrarily and when we started coding we found that we did not have nearly enough time. That, in turn, meant that during the first two weeks our progress was slow and we waited a lot, where we could have designed modules and interfaces, while the Singaporean group was approving the API we had defined. Since we had no planning we suddenly had a lot of code that had to be done in a short amount of time. On top of that we had to communicate to the Singaporean group how to use the service.
 
We had no real concept of how to structure the code in advance, the API that the Singaporeans had approved only covered the web service. We had some idea of what each other were doing, but the communication between us was very limited even between those whose code was directly linked to each others. Each of us was responsible for designing a part of the program instead of designing the program as a whole in group. Where some functionality that could be in more than one place and risked being made both places or not at all. When we discovered that important functionality was missing, it was arbitrary whether it was added to one part or another, or even if it was made in between those. We went well over our first deadline, having underestimated the work that had to be done. The deadline for the whole server coincided the Easter holyday, which made it more difficult for us. Furthermore we had a couple of refactoring, when the persistence turned out not to function as we expected and in order to get rid of code duplication. Up to the last deadline it became very hectic and we ended with a server that did not fulfill every requirements. Even so the result was that the structure became unnecessarily complex. It took much too long, just understanding where an exception was thrown was very hard to see, which in turn made the debugging process frustrating.

After the Singaporean group had their deadline they asked us if we could give them some reflection of the collaboration. It gave us a chance to reflect on our process as well. We realized how far into the semester we were and how close our own deadline was. We saw how much time you could potentially waste, when you do not prepare properly to a given project. At this point we realized that the project had derailed - and something had to be done to get it back on track!
\newpage