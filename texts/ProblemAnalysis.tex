\section{Problem Analysis}
We chose to use our 2\textsuperscript{nd} year project for this project, which was divided into two parts. After the first part of the project, where we built a server while a group in Singapore developed a client using our server, it was evident that we had to do something in order to get the project back on track.

We started out well by making a requirements list, where we listed functional and non-functional requirements of our server. That helped us define what is was we wanted to make and how we wanted to make it. We chose not to use SCRUM, since most of the group felt that it would require more to set up than we would gain from it based on previous experience. Our individual time schedule and short time frame made it difficult to set up sprints, where everybody could participate and were not too short. We did not anticipate how much the lack of planning affected our work, but expected that the time we saved would exceed the lack of planning.

From then on we had no overview of the course of the first part of the project. We decided not to define tasks and parts that needed to be done and give them estimations. We had deadlines on when the different parts of the service had to be fully available, but the dates were set arbitrary and we started coding we found that we did not have nearly enough time. That in turn meant that the first 2 weeks our progress was slow and we waited a lot, where we could have designed modules and interfaces, while the Singaporean group was approving the API we had defined. Since we had no planning we suddenly had a lot of code that had to be done in a short amount of time. On top of that we had to communicate to the Singaporean group how to use the service.
 
We had no real concept of how to structure the code in advance, the API that the Singaporeans had approved only covered the webservice. We had some idea of what each other were doing, but the communication between us was very limited. Those whose code was directly linked to each others . Each of us was responsible for designing the part of the program instead of designing the program as a whole in group. Where some functionality that could be more than one place risked being made both places or not at all. When we disvorvered that an important functionallity was missing, it was arbitrary whether it was added to one part or another, or even if it was made in between those. The result was that the structure became unnecessarily complex. It took much too long, just understanding where an exception was thrown was very hard to see, which made the debugging proces frustating.

After the Singaporean group had their deadline they asked us if we could give them some reflection of the collaboration. It gave us a chance to reflect on our proces aswell. We realized how far into the semester we were and how close our own deadline was. We saw how much time you could potentially waste, when you do not prepare properly to the given project.
\newpage