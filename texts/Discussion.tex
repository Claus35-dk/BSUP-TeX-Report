\section{Discussion}
% 1 page
From the state of the project at the time of writing, with one week left to deadline, our estimations seem to be more or less correct. Important tasks not anticipated from the start have shown, however, meaning that our task estimations largely holds, but the overall estimation of the required time to finish the project does not.

Beside the extra tasks not initially planned for the backlog, unforeseen problems have occurred on the server we developed doing the first part of the project. We had foreseen that some errors probably would arise in regards to functionality on the server, but we had no idea what would fail, how much would fail, how long it would take to debug, or how long it would take to fix the errors if needed.

So far the server errors have been easy to fix, but each time an error has been encountered, time unproportional to the problem itself has been wasted determining whether the error origins from the server or from our newly created client.
This uncertainty is due to poor error reporting facilities, combined with lack of proper testing on the server. The last part only stresses the importance of proper quality assurance.

Though our plans have been subject to delays, the project seems to be back on track. This is suggested by our burndown chart (see figure \ref{burndown} in the appendix) where the lines line up really nicely at the moment. If we are able to continue as currently, we should be able to meet the deadline with all planned functionality implemented for the client.

An important hint at this can also be extracted from the burndown chart, which shows that it is only recently that the project got back on track. This is mainly due to the extra effort which has been needed in order to solve the server problems stated above.
The server and its potential errors should no longer be of much concern since most of the code responsible for communication with the server has been implemented and tested, thereby mainly only leaving tasks concerned with creating a functional GUI.

We believe that we have been able to handle the addition of unforseen tasks only because we choose to reserve much more time than we initially estimated to be required. By comparing
the burndown chart with our timetable its clear that in some periods about twice the time  indicated by the completed backlog tasks were spent completing them. Without extra time to perform unforseen tasks we would currently be far behind schedule.

--------

That we have been able to handle the unforseen tasks while still restoring the project to its...

Beside the code, we also have to write a report for the project. A report that at this time hardly has been started - but have been split into tasks and estimated.
Even though it is primary the GUI components (beside the report) that is left to be done, extra tasks is bound to origin if we look at how is has been so far. Still, we have shown that we are able to catch up with the schedule even though we are behind. Therefore it should be possible for us to meet the deadline. In the worst case we need to downgrade some of the planned functionality or even discard some of them.

It is also appropriate to talk about how we as a group/organization developed during the project. To discuss this, we will evaluate ourselves against the Capability Maturity Model.

The Capability Maturity Model, created by the Software Engineering Institute of Carnegie Mellon University, describes the development process an organization must pass through to be considered mature.

\begin{figure}[t]
  \includegraphics[width=\textwidth]{illustrations/CMM.jpg}
  \caption{Capability Maturity Model. Source: [\textit{http://www.tutorialspoint.com/cmmi/cmmi-maturity-levels.htm}]}
  \label{fig:Capability_Maturity_Model}
\end{figure}

While doing our project, we moved from the initial level of the Capability Maturity Model (while developing the server) to the managed level (during the development of the client) \cite[p. 242]{PM}. We kept to the initial level during the development of the server, by developing as we went along without using well known development methods or initial design (see figure \ref{fig:Capability_Maturity_Model}). It illuminates exactly what we want to show in this report, that using many of these tools gives us a huge advantage of producing a product, while not using any of these tools it is very uncertain any useful product can be made.

----------------------------

To be done...

- Capability Maturity Model improvement

- How well did we manage to rescue the project as seen from the current state of the project? Do it seem like we are succeeding? The report needs much work, we are barely started.
\newpage