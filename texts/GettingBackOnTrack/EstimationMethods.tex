\subsection{Estimation methods}
\subsubsection{Planning Poker}
There exist multiple estimation method, but for this (part of the) project, we have chosen to use Planning Poker, also know as Scrum Poker, to estimate the remaining task for the last part of the project. Planning Poker is a good practice when using SCRUM, and to some sense is using the WBS/PBS paradigm (in SCRUM the backlog).

Planning Poker seemed like a good choice since we all have worked with SCRUM before, but not used Planning Poker, so this was a great opportunity to practice it. Also the fact that the players don't influence each other when estimating, but the game still facilitates shorts discussion is a great advantage. That the game facilitates discussions is the reason for why we chose Planning Poker instead of the Delphi estimation method, which does not. This might be in some cases prove to be a problem. But based on our experience as a group we decided that we all have equal power in the group, and therefor it was very unlikely to become a problem. \\
We could quite quickly ignore the Analogy method since it uses comparison of similar project, which we have not done before, hence it was not an option.

We made playing cards with the numbers 1/2, 1, 2, 3, 5, 8, 13, 21, 34, 55, 89 and "?". The numbers represented man hours and the questionmark represented that the "player" was unsure about the amount of work in the task.
% 3 pages