% !TeX spellcheck = en_US
\subsection{Quality assurance}

So far we have covered the time aspect of managing our project, but have not mentioned the other two aspects expressed by the triple constraint (see figure 1).
The costs aspect does not apply to our project, since all hardware, labor, and work facilities are cost-free, leaving only time-independent costs such as those for printing the final report.
Remaining to be covered is therefore only the quality aspect of managing our project.

To manage the quality of our project, we have devised a procedure for each area we wish manage quality within: To manage the quality of the produced product we have defined a testing strategy; to manage the quality of the code behind the product, we have utilized a tandem of peer- and team leader reviewing.

Our testing strategy dictates that the basic functionality of the business logic is tested using automatic unit tests. The tests should test that the business logic at a basic level works as it is meant to, including that it fails correctly in miscellaneous error situations.
As a compromise between quality and available time, the unit tests should not test trivial border value cases, which otherwise could be identified through equivalence partitioning.
All code related to the GUI of the program should not be tested by unit tests, only by the authoring developer quickly going through the different states the GUI is able to express, checking that all works and looks as expected.
Our testing strategy is by no means exhaustive, but we believe it is adequate.

To ensure the quality of the code, peer reviewing has been applied to the project whenever possible. Through the tools and languages used for the second part of the project is not new to the group members, some are more experienced than others.
We have used this as a benefit, pairing the less experienced with a more experienced group member during development. Additionally, most of the produced code have been reviewed by the most experienced group member which have taken care that any found error or bad design decision was corrected - this may be described as a team leader review \cite[p. 199]{PM}.
Doing these continuous code reviews adds some overhead, but it is an investment: When an error or bad design decision is identified and corrected early on, the time it has taken is usually far shorter than the time it would have taken later, when new layers of code has been based on the bad code and the mistake has accumulated into many more \cite[p. 247]{PM}.

To measure the quality of our product and deem whether it is acceptable, we have decided to acceptance test it against its requirements. If all requirements are met, the quality of the product is acceptable.
