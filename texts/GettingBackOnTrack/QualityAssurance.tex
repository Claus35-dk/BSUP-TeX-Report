\subsection{Quality assurance}
To be able to measure the quality of our client we use the requirements we set to the client along with a test strategy to ensure as little errors as possible and the methods descriped earlier in this chapter. We move from the initial level of Capability Maturity Model while developing the server to the managed level during the development of the client. We kept to the initial level during the development of the server, by developing as we went along.

The usage of a test strategy, increases the likelyhood of finding errors early and correcting them, thereby minimizing the amount of time on debugging. When we define the test criteria, combined with the product requirements, we get one way of measurering quality.

 If the product fulfills the requirements and pass the tests, depending on the tests, its reliability has been proved as well. Which is one of CISQ's 5 major desirable characteristics. The other 4 are; efficiency, security, maintainablility and size. The efficiency parameter comes partly from our planning and software design and partly from our implementation.
 Our design assumes a minimal amount of the security on the client side, most of the security is done by the server.
 The client is maintainable, we have a well documented and responsibility driven product, making maintenance much easier. 
 Size is not a quality indicator per se, but can be used to asses the time needed to complete the product. It can however be used determined if we have made it too complicated.
 