\subsection{Method of development}
% 3 pages
We like the SCRUM development method, but have decided only to do some of the SCRUM activities.

A SCRUM activity we have chosen to exclude among others is Daily SCRUM.
When you do Daily SCRUM the development team, the product owner and the SCRUM-master is updated with progress and set-backs, which generally is a good thing \cite[p. ?]{?}.
We are a small group of only 5 people, however, and we do not have people to fill out all roles used in proper SCRUM, such as product owner and stake holders - we are only able to represent the development team.
As development will be done with the group assembled, we see no need for spending 15 minutes every meeting updating each other, as we already do this ad hoc when we are working.
The API we have defined and the SCRUM backlog should be enough to make the progress visible to all team members.

[ TODO: Insert SCRUM figure, displaying the various parts of a typical SCRUM process ]

As just mentioned, we have decided to use a backlog, which is another SCRUM artifact.
The backlog is a prioritized list of items, where each item describes a program feature called a `story', possibly with sub-stories or tasks that needs to be completed \cite[p. ?]{?}. Traditionally the product owner defines the order of the items, putting first the features he wants implemented first \cite[p. ?]{?}.
We have chosen to use a backlog as it ensures that the most vital functionality is implemented over less important, nice-to-have features. We will also benefit from producing the backlog itself, as we will be forced into enumerating all features to implement, giving us an overview of the work and ensuring that no big task suddenly pops up later.
Our complete backlog can be viewed in \acref{backlog}.

We have chosen not to divide our project into sprints, even though they are a core foundation of SCRUM.
Sprints are great for projects with a longer time spans than ours, as it divides the project into smaller iterations, each producing valuable functionality able to function on its own, making it easier keep focus \cite[p. ?]{?}.
After every sprint, the SCRUM team do a review, a retrospective, and plan which backlog items to implement during for the next sprint \cite[p. ?]{?}.
Were we to divide our project into sprints, it would be very short ones, and it would not be worth the overhead from managing sprint backlogs and during reviews and retrospectives.

[ TODO: Describe SCRUM Task Board and our use ]
Our task board can be viewed in \acref{taskboard}.

[ TODO: Describe SCRUM Burndown Chart and our use ]
Our burndown chart can be viewed in \acref{burndown}.

We will use scrumwise.com as our SCRUM tool to reduce the overhead from the above methods we have chosen to use. The overhead will be reduced, since artifacts, such as the burndown chart and the task board, automatically will be updated.
Because the various artifacts automatically are updated, we will easily be able to monitor our progress and maintain our overview. Should our work not progress as estimated and planned, we will be able to spot it immediately, and may then pause our work to reevaluate our plans, possible doing re-estimation of some or all of the tasks.