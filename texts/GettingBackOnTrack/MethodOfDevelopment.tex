\subsection{Method of development}
% 3 pages
We like the Scrum development method, but we decided not to do full-on Scrum, but to do some Scrum activities. Daily Scrum is a great activity. When you do Daily Scrum the Development team, the Product owner and the Scrum master is held updated with progress and set-backs. We have chosen not to do Daily Scrum, because we are a small group of only 5 people, and we don't have the roles used for proper Scrum (Product owner, Scrum master, etc.). We are only the Development team and we have the API and Backlog to as tools to keep each other up to date, and there for Daily Scrum is not worth the overhead.

We chose to use another Scrum artifact: The backlog. The backlog is a list of items. Each item describes a `story', possibly with sub stories or tasks, that needs to be completed. An item on a backlog could be ``Implement controller class'' or ``Write problem statement''. Our complete backlog can be viewed in \acref{backlog}.

We chose not to divide our project into Sprints. Sprints are great for projects with a longer time span than ours, as it divides the project into smaller parts which makes it almost iterative. Sprints also function as part goals. After every Sprint, the Scrum team do a Review, do a Retrospective and plan the Backlog items for the next Sprint. Were we to divide our project into Sprints, it would be very short ones and it would not be worth the overhead.

We used Scrumwise.com as our Scrum tool. Scrum task board and burndown chart were by default given to us without any extra effort. We chose to use this "free service" to regularly check up on our progress, and to see if we had to stop and reevaluate anything - like the estimation for some or all of the tasks. It helped us gain overview doing the project. Our task board can be viewed in \acref{taskboard}. Our burndown chart can be viewed in \acref{burndown}.