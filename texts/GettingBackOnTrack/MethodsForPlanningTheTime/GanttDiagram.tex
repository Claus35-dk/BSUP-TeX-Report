\subsubsection{GANTT diagram}
A GANTT diagram can be an extremely efficient tool for planning in larger groups as it gives an excellent and very concrete overview of when activities should start, be finished, and how they should be carried out in relation to each other.

For our project we need fine-grained planning of each task with the agility to quickly switch the order if deemed necessary, which we do not feel a GANTT diagram is able to provide us - at least not with the tools we know of and have access to.

If changes to the order of tasks are common, the overhead to keep the GANTT diagram updated will be too large, especially because we already have decided to use a dependency network diagram, which states the same activity relationships as a GANTT diagram, albeit not with the same compactness.

A GANTT diagram might be the right choice with a more plan-driven development method, but for this project we do not believe it to add value enough for its overhead. We are short on time, so any time we are able to free from unneeded activities should be so and used for more important things, such as quality assurance.