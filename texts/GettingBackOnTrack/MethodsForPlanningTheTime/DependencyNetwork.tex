\subsubsection{Dependency network with activity durations}
After planning all our group meetings we decided to create a dependency network with activity durations. We had several reasons for doing this:

\begin{itemize}
	\item We wanted to make sure that everyone always knew what to do. A dependency network is great for this purpose, as everyone can easily see which elements must precede others.
	\item We wanted to clearly identify the critical path. This allowed us to make sure that minimum slack was applied to the path and to keep the project within its designated time frame.
	\item To be able to see available slack of backlog items. This helped us very much in determining which items to complete first.
\end{itemize}

Our rather large dependency network diagram may be found in the appendix (see figure 8).
Some important parts to point out that we learned from this network:

\begin{itemize}
	\item It shows that much of the report is able to be written before any implementation of the client is done. Though logical, it was not something we had considered before the network surprisingly revealed it.
	
	\item It shows that we have to begin programming the controllers/models before views. Additionally it shows that not all controllers/models need to be finished before the first views can be implemented. It also shows exactly what functionality is required by each view.
\end{itemize}

Although the dependency network indicates that all work can be done in 87 hours, this is not entirely true, as more tasks has been parallelized than the group is able to commit workers. If the project should be completed in 87 hours, one or more group members would have to work on two tasks simultaneously, while still performing the tasks as efficient as if they had been completed in serial. Of course this is highly unrealistic.

While totally acceptable, it means that the network cannot be used to measure time left. When determining what task to complete next, we have to take into account the delay on the parallel processes requiring more workers than available group members.

At the time of writing, the network has helped us a lot in our work, both when the backlog was to be prioritized, but also as a visual representation of the tasks from which our progress could be seen.