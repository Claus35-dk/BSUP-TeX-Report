\subsubsection{Dependency network with activity durations}
After planning all our group meetings we decided to create a dependency network with activity durations. We have several reasons for doing this
\begin{itemize}
	\item We wanted to make sure that everyone always knew what to do. A dependency network is great for this purpose, as everyone can easily see which elements must precede others.
	\item We wanted to clearly identify the critical path. This allow us to make sure that minimum slack is applied to the path and to keep project on deadline.
	\item To be able to see available slack of backlog items. This helped us very much in determining which items to complete first.
\end{itemize}

\begin{figure}[H]
%\includegraphics[width=\textwidth,natwidth=2157,natheight=1336]{illustrations/backlog.png}
  \caption{Dependency network with activity durations}
  \label{dependency network}
\end{figure}

\cref{dependency network} shows our dependency network.
Some important parts to point out that we learned from this network:
\begin{itemize}
	\item It shows that testing WCF controller can be pushed all the way to the 73\textsuperscript{rd} hour. We suspected that this part was not that important, but the network really confirmed us in this suspicion.
	\item It shows that we have to begin programming the controllers/models before views. It also shows that not all controllers/models need to be finished before the first views can be implemented. It shows exactly which controllers/models that needs to be finished.
\end{itemize}

The network has helped us a lot in our work, and we have returned to it many times throughout the project.