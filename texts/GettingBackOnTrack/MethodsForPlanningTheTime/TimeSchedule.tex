\subsubsection{Time schedule}
To be able to plan our time efficiently, we had to find out how much time we had at our disposal. Our way of doing this was to create a custom spreadsheet, containing all the dates until deadline. For each date a field for each person was present. In this field the hours available should be entered. This way it was easy for us to get an overview of when we all were available. This helped us a lot in planning our meetings.

Generally, we have decided, that if three or more group-members are available, a meeting should be planned. We would rather be done with everything before time, than only plan the hours that we needed to complete the backlog items

-- Insert diagram here --

Figure x shows our custom spreadsheet using this custom method. We have really had great benefit from doing it this way, as planning group meetings just was instant, without too much communication back and forth.

Whenever three or more group members had placed available hours in the same timespan on a specific date, we would plan a group meeting at that specific time. This meant that we had more group meetings than needed, but this also helped us meet our deadline on time.