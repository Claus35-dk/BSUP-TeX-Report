\subsubsection{Time schedule}
To be able to plan our time efficiently, we had to find out how much time we had at our disposal. Our way of doing this was to create a custom spreadsheet, containing all the dates until deadline. For each date a field for each person was present. In this field the hours available were entered. This way it was easy for us to get an overview of when we all were available. This helped us a lot in planning our meetings.

Generally, we decided, that if three or more group-members were available, a meeting would be planned. We would rather plan too many meetings and be done with everything early, than plan the exact hours needed, to complete the backlog items, over the entire timespan and risk time pressure due to unforeseen tasks or delays.

\begin{figure}[H]
%\includegraphics[width=\textwidth,natwidth=2157,natheight=1336]{illustrations/backlog.png}
  \caption{Time schedule}
  \label{time schedule}
\end{figure}

\cref{time schedule} shows our custom spreadsheet using our custom method. We have had great benefit from doing it this way, as planning group meetings was instantaneous, without too much discussion back and forth.